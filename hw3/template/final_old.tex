%!!!!!!!!!!!!!!!!!!!!!!!!!!!!!!!!!!!!!!!!!!!!!!!!!!!!!!!!!!!!!!!!!!!!!!!!!!!!!!
%!NOTE: This example file has been prepared according to the University of
%!      Hawaii Style & Policy Manual for Theses and Dissertations dated
%!      "Revised September 2010". If you have one with a later date, you may
%!      need to make revisions to this document as well. In any event, making
%!      sure your thesis complies with Graduate Education guidelines is
%!      ultimately your responsibility. Caveat LaTeXtor. :)
%!!!!!!!!!!!!!!!!!!!!!!!!!!!!!!!!!!!!!!!!!!!!!!!!!!!!!!!!!!!!!!!!!!!!!!!!!!!!!!

%% The options are (you can only choose one from each group):
%%
%% 10pt, 11pt, 12pt: chooses the point size for the document. "11pt" is the
%%                   default.
%%
%% oneside, twoside: whether you want your document onesided or twosided. Note
%%                   that twosided is not guaranteed to work, and style
%%                   guidelines prohibit double sided printouts on final
%%                   copy. "oneside" is the default.
%%
%% draft, final: when printing drafts you can save a lot of paper by using the
%%               "draft" option. It switches to single spacing, displays overful
%%               hboxes with a black box, prints a version number on title page 
%%               and omits signature page. Of course for the final copy make
%%               sure to use the "final" option! "final" is the default.
%%
%% thesis, dissertation: switches between the style for a master's thesis and a 
%%                       Ph.D. dissertation. The differences are fairly minor
%%                       and limited to the front matter. "thesis" is the
%%                       default.
%%
%% actual, proposal: switches between actual document and proposal mode. In
%%                   proposal mode: the title page is simplified and the
%%                   version number is always printed.
%%
%%% Load the new uhthesis document class
%\documentclass[12pt]{article}
\documentclass[conference]{IEEEtrans}

\usepackage[table,xcdraw]{xcolor}
\usepackage{setspace}
\usepackage{paralist}
%\usepackage{amsmath}
\usepackage{bm}
%\usepackage{subfigure}
\usepackage{cite}
\usepackage{multirow}
\usepackage{todonotes}
\usepackage{umoline}
\usepackage{xspace}
\usepackage{soul}
%\usepackage{gensymb}
\usepackage[euler]{textgreek}
\usepackage{array}
\usepackage{tabu}
\usepackage{tabularx}
\usepackage{subcaption}
\usepackage{amsmath}
\usepackage[symbol]{footmisc}
%\usepackage{table}
%\usepackage{xcolor}
%\usepackage{xcdraw}

%%% Custom Macros %%%
\newcommand\tab[1][1cm]{\hspace*{#1}}
\newcommand{\LL}[2][inline]{\todo[color=red!50,#1]{\sf \textbf{LL:} #2}\xspace}
\newcommand{\HC}[2][inline]{\todo[color=blue!50,#1]{\sf \textbf{HC:} #2}\xspace}
\newcommand{\gridsize}{\texttt{GRID\_SIZE\xspace}}
\newcommand{\celsius}{$^\circ$C\xspace}
\newcommand{\hotspot}{Hotspot\xspace}
\newcommand{\crosstalk}{crosstalk\xspace}
\newcommand{\Crosstalk}{Crosstalk\xspace}
%\newcommand{\alpha}{\textalpha }
%%% Load some useful packages:
%% New LaTeX2e graphics support
\usepackage{graphicx}
%% Package to linebreak URLs in a sane manner.
\usepackage{url}
\usepackage{float}
\usepackage{paralist}
\usepackage{listings}
\iffalse
\lstset{frame=tb,
  language=Python,
  aboveskip=3mm,
  belowskip=3mm,
  showstringspaces=false,
  columns=flexible,
  basicstyle={\small\ttfamily},
  numbers=left,
  numberstyle=\tiny\color{black},
  keywordstyle=\color{blue},
  commentstyle=\color{dkgreen},
  stringstyle=\color{red},
  breaklines=true,
  breakatwhitespace=true,
  tabsize=3
}
\usepackage{geometry} % Required for adjusting page dimensions and margins

\geometry{
        paper=a4paper, % Paper size, change to letterpaper for US letter size
        top=2.5cm, % Top margin
        bottom=3cm, % Bottom margin
        left=2.5cm, % Left margin
        right=2.5cm, % Right margin
        headheight=10pt, % Header height
        footskip=1.5cm, % Space from the bottom margin to the baseline of the footer
        headsep=1.2cm, % Space from the top margin to the baseline of the header
        %showframe, % Uncomment to show how the type block is set on the page
}

\usepackage{lmodern}%get scalable font
%\usepackage{showframe}
\usepackage{titling}
\pretitle{\begin{center}\fontsize{18bp}{14bp}\selectfont}
\posttitle{\par\end{center}}
\preauthor{\begin{center}\fontsize{14bp}{14bp}\selectfont}
\postauthor{\par\end{center}\vspace{14bp}}
\predate{}
\date{}
\postdate{}
\fi
\title{ICS674 Project Proposal}

\author{Lambert Leong}


\begin{document}

%\textbf{ICS674 Mini Project}
%\\    Lambert Leong

\maketitle
\iffalse
Designing the best integrated 3D multi-chip system which optimizes temperature,
chip frequencies, and inter-chip network topology is an NP-hard multi-objective
optimization problem.  The use of heuristics to solves this problem is
necessary to arrive at good layout geometries in a tractable amount of time.
The Hotspot v6.0 heat simulator was used in previous work to accurately measure
the operating temperature of the integrated 3D multi-chip system however, it is
a computationally expensive and a lengthy process.  Thusly, Hotspot's lengthy runtime
limited the heuristics we were able to employ for exploring different layout
geometries.  
\fi

\section{Approach}

A trade-off exists between temperature and chip power and also between
temperature and the inter-chip network topology.  By transitivity there is also
a trade off between chip power and inter-chip network topology.  When designing
the best layouts for integrated 3D multi-chip systems we deal with these
trade-offs in that we strive to achieve a layout that yields a system with a
high operating chip power and frequency and good inter-chip network topology that
can operate under a given temperature constraint.  This is a constrained
multi-objective optimization problem which is NP-hard.  In this project I employ
the use of evolutionary computational techniques, in the form of a genetic
algorithm, to explore the vast layout design space and generate the best
layouts for a given layout size.

\subsection{Acceleration of Heat Objective Function}
Implementations of a genetic algorithm would require multiple calls to the slow
Hotspot simulator for an individuals fitness evaluations via some objective
function. Thusly, given the current slow state of our heat evaluator, it would
be difficult to implement a genetic heuristic that generates good quality
layouts in a tractable amount of time.  An approach, which we are proposing in
this work, would be to develop a method of deriving a fast approximation of the growing
layout temperature.  My previous work has resulted in a sizable sample set of
layout configurations and corresponding temperature values.  With this sample
set we attempt to train a machine learning regression to learn the relationship
between the layout configuration and output temperature.  This trained
regression will help to return fast temperature approximations when given a
particular layout configurations.  As a result we will be able to implement a
genetic algorithm which will not be bottlenecked by slow temperature
evaluations.  The Hotspot simulator, while slow, is extremely accurate and thus
we plan to call it for a final temperature evaluation the the final layout
generated via a genetic algorithm.  

\subsection{Dimensionality Reduction}
As mentioned previously, a trade-off exists between the layout temperature and
the tightness of the network and tighter networks correlate to higher
temperatures.  Another possible way of characterizing a tight network is layout density.
Denser 3D layouts, like tight network topologies, correlate to higher potential
layout temperatures.  It may be possible to derive a simpler heat approximation
that leverages the, supposed, trade-off relationship between temperature and
layout density.  In addition, a relationship does exists between density, power, and
temperature and thus power needs to be factored into this simpler heat
approximation method. As a result, we can determine a simple heat approximation
by calculating density first and multiplying it by the chip power.  We deem
the result of the product between density and power the ``heat score" which is
now a single objective measurement which represents two data dimensions, density
and power.  

\section{Target Results}

Results from my previous work showed that our random greedy heuristic which used
a (1, $\lambda$)-ES generated layout configurations that were better than the
checkerboard layout.  However, the network characteristics for both the
checkerboard and our (1, $\lambda$)-ES heuristic converged to be identical as
the layout size or number of chips within the layout increased.  In these
instances, where a tie in network characteristics exists, our heuristic layouts
were deemed superior by virtue of the computational power of the system layout.
Heuristic generated layout were able to run at a higher chip power even though
the network characteristics were identical.   

Ideal results would include layouts that beat the checkerboard layout with
respect to all the objective (network, power, and temperature) for any size
layout. We hope to achieve these ideal results with the implementation of a
genetic algorithm.  Successful implementation and execution of a genetic
algorithm for layout generation will be measured according to the three
objectives.  Additionally, we will evaluate the genetic algorithms approach
against checkerboard construction and our previous (1, $\lambda$)-ES via direct
comparison of the resulting layout characteristics. Our target results would
demonstrate the genetic algorithms ability to generate layouts that can operate
at higher chip powers than the checkerboard, just as the (1, $\lambda$)-ES
generated layouts, as well as a better inter-chip network topology unlike our
previous heuristic that could only match it.    

\subsection{Challenges}

Chip positions are represented, in simulation, as a 3D graph point of the bottom
left most edge of the chip and these points are stored in an array.  In reality,
these graph points represent a physical chip the occupy space which physically
constrains placement of chips, as to avoid collisions. This physical constraint
may make the implementation of a variation operator such as cross-over difficult.
A simple cross-over would involve splitting one layout and recombining it with
another part of a layout.  While this seems simple in theory, the physical
constraints of dealing with actual hardware chips and still attempting to
maintain connectivity between chips is difficult in practice.   

\section{Resources}

* \emph{The following references pertain to the background and previous work regarding
integrated 3D multi-chip systems.  Evolutionary resources, aside from the textbook and resources provided by Professor Altenberg, are being gathered.}

\nocite{*}
\bibliography{biblio}
%% Use this for an alphabetically organized bibliography
\bibliographystyle{plain}



%have to go the angle of fast approximations for the heat evals because heat
%needs to be evaluated when levels are changed, therefore hotspot called




\iffalse
For my evolutionary computation mini project I implemented a genetic algorithm
that matches an input string.  Code implementations are written in Python2.7.

\section{Search Space}

The search space consists of all alphabet characters, upper and lower case.
This can be seen in Listing~\ref{lst:search_space} where we randomly fill the
search strings with letters by calling \texttt{string.letters} from Pythons
string module.

\begin{lstlisting}[caption = {Search space is all letter characters, upper and
lower case }, label = {lst:search_space}]
self.string = ''.join(random.choice(string.letters) for _ in xrange(length))
\end{lstlisting}

\section{Variation Operator}

Two variation operator are implemented which include crossover and mutation.

\subsection{Crossover}

To perform crossover we we randomly select two parent strings from the previous
generation, seen in lines 4\&5 of Listing~\ref{lst:cross}.  I randomly select an
integer which corresponds to the array index at which the parent string is split
for each child to inherit, seen i line 8.  Child 1 gets the char from the first
half from parent 1 and the second half from parent 2.  Child 2 gets the first
half from parent 2 and second half from parent 1.

\begin{lstlisting}[caption = {Crossover Function}, label = {lst:cross}]
def crossover(individuals):
        offspring = []
        for _ in xrange((population - len(individuals))/2):
                parent1 = random.choice(individuals)
                parent2 = random.choice(individuals)
                child1 = Individual(in_str_len)
                child2 = Individual(in_str_len)
                split = random.randint(0, in_str_len)
                child1.string = parent1.string[0:split] + parent2.string[split:in_str_len]
                child2.string = parent2.string[0:split] + parent1.string[split:in_str_len]
                offspring.append(child1)
                offspring.append(child2)
        individuals.extend(offspring)
        return individuals
\end{lstlisting}

\subsection{Mutation}

Mutation occurs in the for of switching out a character in a search string with
a random letter.  All strings in the population are susceptible to mutation and
multiple mutation can occur in a individual string.  The mutation rate is 5\% as
indicated in line 4 of Listing~\ref{lst:mutation}.

\begin{lstlisting}[caption = {Mutation Function}, label={lst:mutation}]
def mutation(individuals):
        for individual in individuals:
                for i, param in enumerate(individual.string):
                        if random.uniform(0.0, 1.0) <= 0.05:
                                individual.string = individual.string[0:i] + random.choice(string.letters) + individual.string[i+1:in_str_len]
        return individuals
\end{lstlisting}


\section{Selection Operator}

Individual strings are sorted based upon their fitness scores, as seen in
Listing~\ref{lst:selection}.  Only the top 20\% of the individual strings are
kept for each generation, which can be seen in line 6.  Lines 3-5 are for
graphing purposes and not part of the selection function.

\begin{lstlisting}[caption = {Selection Function}, label = {lst:selection}]
def selection(individuals):
        individuals = sorted(individuals, key=lambda individual: individual.fitness, reverse=True)
        max_fit.append(max(individuals, key=lambda individual: individual.fitness).fitness)
        min_fit.append(min(individuals, key=lambda individual: individual.fitness).fitness)
        avg_fit.append(float(sum(i.fitness for i in individuals)//len(individuals)))
        individuals = individuals[:int(0.2*len(individuals))]
        return individuals
\end{lstlisting}

\section{Termination Criterion}

Listing~\ref{lst:term} displays the termination code.  The algorithm terminates
when it has reached the last generation, seen i line 1, or when the fitness is
100, seen in line 7.  Fitness scores are out of a 100 and a score of 100 equates
to an identical and successful string match.

\begin{lstlisting}[caption={Termination}, label={lst:term}]
 for generation in xrange(generations):
                generation_list.append(generation)
                individuals = fitness(individuals)
                individuals = selection(individuals)
                individuals = crossover(individuals)
                individuals = mutation(individuals)
                if any(individual.fitness >= 100 for individual in individuals):
                        found = True
                        break
\end{lstlisting}

\section{Objective Fuction}

Figure~\ref{fig:fitness} indicates the max, min, and average fitness scores in
each generation.  During this run, the string was matched in a little over 350
generation.  The blue line reaches a fitness score of 100 which is the max
fitness score and indicates a successful match of the input string which was
``HelloWorld".  Fitness scores were calculated with the objective function shown
in Listing~\ref{lst:fit}

\begin{figure}[H]
        \centering
        \includegraphics[width=.8\textwidth]{figures/Figure_3.png}
        \caption{Max, average, and min fitness values of each individual string
for each generation}
        \label{fig:fitness}
\end{figure}

As mentioned above, fitness scores are out of 100 and they are calculated as
percentages.  The \texttt{total} is given the value of twice the length of the
string, seen in line 3.  If the search/individual string contains a letter that
is also contained in the input string, the fitness \texttt{score} is incremented
by 1.  This process occurs in lines 9-14.  Once a letter from the search string
is found to be in the input string, that letter is removed from the input string
for future searches as to avoid duplicates and unwanted fitness score
increases.  
\\ \\
Scores are also incremented if the
search string contains the right letter in the right location or at the same
index as the input string, which can be seen in lines 5-7. The score is then
divided by the total and multiplied by 100 to get the resulting fitness score,
shown in line 15.

\begin{minipage}{\textwidth}

\begin{lstlisting}[caption = {Fitness Function}, label = {lst:fit}]
def fitness(individuals):
        for individual in individuals:
                total = len(in_str)*2
                score = 0
                for i, letter in enumerate(individual.string):
                                if in_str[i] == letter:
                                        score += 1
                compare_str = in_str
                for a_char in individual.string:
                        for i, in_char in enumerate(compare_str):
                                if a_char == in_char:
                                        score += 1
                                        compare_str = compare_str[:i]+compare_str[i+1:]
                                        break
                individual.fitness = int((float(score)/float(total))*100)                          
        return individuals
\end{lstlisting}
\end{minipage}
\fi
\end{document}
