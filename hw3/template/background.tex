\section{Throughchip Interface (TCI)}
\label{sec:background}

% Introduce TCI
Through chip interface (TCI) is a wireless technology that allows for inter-chip
communication via inductive coupling.  TCI has many advantages over traditional,
wired, communication technologies in that it prefabrication of chips is not
required.  Wire bonding and throuch scilicon via (TSV) communication
technologies require the running of wires through the chip, which affects the the
processing unit and eliminates the possibility of having a transiter in that
location.  Inductors sit on top of chips and TCI does not disrupt or require
modifications to existting chips. An example of TCI is shown in
Figure~\ref{fig:tci}.

\begin{figure}[h]
        \centering
        \includegraphics[width=70mm]{TCI.jpg}
	\caption{Connecting partially overlapping chips using inductors.
(Courtesy of Prof. Michihiro Koibuchi, National Informatics Institute, Japan.)}
        \label{fig:tci}
\end{figure}

 Inductive-coupling offers high speed (e.g., 8
Tbps~\cite{MiuraISSCC2010}), low power (e.g., 0.14 pJ/b~\cite{MiuraISSCC2007})
due in part to the removal of electrostatic discharge (ESD) protection devices,
high integration (e.g., a 128-die NAND stack~\cite{KurodaISSCC2010}), and
applicability to an increasing range of computing systems including a
reconfigurable processor~\cite{MiraMICRO2013}.  All the features afforded by TCI
allows for the exploration of the vast chip layout design space.

%    crosstalk
While TCI is a favorable communication technology for constructing multi-chip
layouts, it presents a unique design challenge.  Interference, known as
crosstalk can occur if wireless communication regions are too close to one
another. Crosstalk occurs between vertical links that are physically located
directly above or below each other.  Interference as a result of crosstalk
negatively impacts inter-chip communication, which in turns can dramatically
reduce application performance.  Crosstalk can be avoided altogether by doubling
the footprint of the induction area~\cite{Kuroda2007,ISSCC2009}.  Constructing 3-D layouts with the constraint that no two
vertical links can be placed directly above each other. 

% previous layout configureation
The checkerboard layout can utilizes TCI for communication as shown below in
Figure~\ref{fig:init_checkerboard}.  To avoid crosstalk, the checkerboard is
limited to only two levels of chip stacking.  A checkerboard layout geometry is
easy to conceptualize and it scales with growing number of chips.  Thusly, the checkerboard layout will serve as our control and we will compare our GA generated layouts to it.

\begin{figure}[h]
        \centering
        \includegraphics[width=60mm]{hv-layout.eps}
        \caption{Checkerboard layout. (Courtesy of Prof. Michihiro
Koibuchi, National Informatics Institute, Japan.)}
        \label{fig:init_checkerboard}
\end{figure}

% prevous work showed similarity towards checerboard
%   is this the best network? 

\iffalse
3D multi-chip stacking of microprocessor chips is an idea that is motivated by
the 3D stacking of DRAM.  Stacking of DRAM has lead to high bandwidth
performance benefits.  Previous work in 3D stacking of microprocessor chips has
explored two primary layout geometries which includes the stack and checkerboard
layout.  The stack layout consists of chips stacked vertically on top of each
other while the checkerboard layout consists of chips that only overlap each
other at the corners so that when viewed from above it resembles a
checkerboard.  The checkerboard layout is a more advantageous layout geometry in
comparison to the stack due to it's ability to dissipate heat which is afforded
by its sparser configuration.  The checkerboard layout also scales.

In my previous work we employed heuristics to try and build integrated 3D
multi-chip systems that beat the checkerboard with respect to better network
topology and higher chip powers given a particular number of chips and
temperature constraint.  During this work, we were severely limited in the types
of heuristic we could use due a massive simulation bottleneck.  Heat evaluations
of the layout during construction needed to be performed empirically by the
Hotspot v6.0 simulator which has a lengthy runtime.  In this previous, we
implemented a (1, $\lambda$)-evolutionary strategy (ES), despite the Hotspot
bottleneck
and produced good layouts~\cite{MiuraISSCC2010}.
%\iffalse
Designing the best integrated 3D multi-chip system which optimizes temperature,
chip frequencies, and inter-chip network topology is an NP-hard multi-objective
optimization problem.  The use of heuristics to solves this problem is
necessary to arrive at good layout geometries in a tractable amount of time.
The Hotspot v6.0 heat simulator was used in previous work to accurately measure
the operating temperature of the integrated 3D multi-chip system however, it is
a computationally expensive and a lengthy process.  Thusly, Hotspot's lengthy
runtime
limited the heuristics we were able to employ for exploring different layout
geometries.
\fi

