\iffalse
Designing the best integrated 3D multi-chip system which optimizes temperature,
chip frequencies, and inter-chip network topology is an NP-hard multi-objective
optimization problem.  The use of heuristics to solves this problem is
necessary to arrive at good layout geometries in a tractable amount of time.
The Hotspot v6.0 heat simulator was used in previous work to accurately measure
the operating temperature of the integrated 3D multi-chip system however, it is
a computationally expensive and a lengthy process.  Thusly, Hotspot's lengthy runtime
limited the heuristics we were able to employ for exploring different layout
geometries.  
\fi

\section{Approach}

A trade-off exists between temperature and chip power and also between
temperature and the inter-chip network topology.  By transitivity there is also
a trade off between chip power and inter-chip network topology.  When designing
the best layouts for integrated 3D multi-chip systems we deal with these
trade-offs in that we strive to achieve a layout that yields a system with a
high operating chip power and frequency and good inter-chip network topology that
can operate under a given temperature constraint.  This is a constrained
multi-objective optimization problem which is NP-hard.  In this project I employ
the use of evolutionary computational techniques, in the form of a genetic
algorithm, to explore the vast layout design space and generate the best
layouts for a given layout size.

\subsection{Acceleration of Heat Objective Function}
Implementations of a genetic algorithm would require multiple calls to the slow
Hotspot simulator for an individuals fitness evaluations via some objective
function. Thusly, given the current slow state of our heat evaluator, it would
be difficult to implement a genetic heuristic that generates good quality
layouts in a tractable amount of time.  An approach, which we are proposing in
this work, would be to develop a method of deriving a fast approximation of the growing
layout temperature.  My previous work has resulted in a sizable sample set of
layout configurations and corresponding temperature values.  With this sample
set we attempt to train a machine learning regression to learn the relationship
between the layout configuration and output temperature.  This trained
regression will help to return fast temperature approximations when given a
particular layout configurations.  As a result we will be able to implement a
genetic algorithm which will not be bottlenecked by slow temperature
evaluations.  The Hotspot simulator, while slow, is extremely accurate and thus
we plan to call it for a final temperature evaluation the the final layout
generated via a genetic algorithm.  

\subsection{Dimensionality Reduction}
As mentioned previously, a trade-off exists between the layout temperature and
the tightness of the network and tighter networks correlate to higher
temperatures.  Another possible way of characterizing a tight network is layout density.
Denser 3D layouts, like tight network topologies, correlate to higher potential
layout temperatures.  It may be possible to derive a simpler heat approximation
that leverages the, supposed, trade-off relationship between temperature and
layout density.  In addition, a relationship does exists between density, power, and
temperature and thus power needs to be factored into this simpler heat
approximation method. As a result, we can determine a simple heat approximation
by calculating density first and multiplying it by the chip power.  We deem
the result of the product between density and power the ``heat score" which is
now a single objective measurement which represents two data dimensions, density
and power.  

\section{Target Results}

Results from my previous work showed that our random greedy heuristic which used
a (1, $\lambda$)-ES generated layout configurations that were better than the
checkerboard layout.  However, the network characteristics for both the
checkerboard and our (1, $\lambda$)-ES heuristic converged to be identical as
the layout size or number of chips within the layout increased.  In these
instances, where a tie in network characteristics exists, our heuristic layouts
were deemed superior by virtue of the computational power of the system layout.
Heuristic generated layout were able to run at a higher chip power even though
the network characteristics were identical.   

Ideal results would include layouts that beat the checkerboard layout with
respect to all the objective (network, power, and temperature) for any size
layout. We hope to achieve these ideal results with the implementation of a
genetic algorithm.  Successful implementation and execution of a genetic
algorithm for layout generation will be measured according to the three
objectives.  Additionally, we will evaluate the genetic algorithms approach
against checkerboard construction and our previous (1, $\lambda$)-ES via direct
comparison of the resulting layout characteristics. Our target results would
demonstrate the genetic algorithms ability to generate layouts that can operate
at higher chip powers than the checkerboard, just as the (1, $\lambda$)-ES
generated layouts, as well as a better inter-chip network topology unlike our
previous heuristic that could only match it.    

\subsection{Challenges}

Chip positions are represented, in simulation, as a 3D graph point of the bottom
left most edge of the chip and these points are stored in an array.  In reality,
these graph points represent a physical chip the occupy space which physically
constrains placement of chips, as to avoid collisions. This physical constraint
may make the implementation of a variation operator such as cross-over difficult.
A simple cross-over would involve splitting one layout and recombining it with
another part of a layout.  While this seems simple in theory, the physical
constraints of dealing with actual hardware chips and still attempting to
maintain connectivity between chips is difficult in practice.   

\section{Resources}

* \emph{The following references pertain to the background and previous work regarding
integrated 3D multi-chip systems.  Evolutionary resources, aside from the textbook and resources provided by Professor Altenberg, are being gathered.}

\nocite{*}
\bibliography{biblio}
%% Use this for an alphabetically organized bibliography
\bibliographystyle{plain}



%have to go the angle of fast approximations for the heat evals because heat
%needs to be evaluated when levels are changed, therefore hotspot called



